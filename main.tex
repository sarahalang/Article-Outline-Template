\documentclass{article}
\usepackage[utf8]{inputenc}
\usepackage[margin=19mm]{geometry}

%--------------------------------------------------------------------------------------------
% VARIOUS PACKAGES  
%--------------------------------------------------------------------------------------------
\usepackage{xcolor}
\usepackage{lipsum}

%--------------------------------------------------------------------------------------------
% TCOLORBOX
%--------------------------------------------------------------------------------------------
\usepackage{tcolorbox}
\tcbset{colback=white,colframe=black!90}



%--------------------------------------------------------------------------------------------
% CHANGE SECTION TITLES
%--------------------------------------------------------------------------------------------
\usepackage{titlesec}
\titleformat{\section}{\huge\sffamily\bfseries}{}{0em}{}[]


%--------------------------------------------------------------------------------------------
% FONT, QUOTE
%--------------------------------------------------------------------------------------------
\usepackage{fontenc}
\usepackage[default]{raleway}
\usepackage{fontawesome}
\renewenvironment{quote}
               {\list{\faQuoteLeft\phantom{ }}{\rightmargin\leftmargin}%
                \item\relax\scriptsize\ignorespaces}
               {\unskip\unskip\phantom{xx}\faQuoteRight\endlist}
               



%--------------------------------------------------------------------------------------------
% FOOTER NINJA-LOGO
%--------------------------------------------------------------------------------------------              
\usepackage{tikz}
\usepackage{graphicx}
\newcommand{\roundpic}[1]{\begin{figure}[h!]\tikz  \draw [path picture={ \node at (path picture bounding box.center){\includegraphics[height=2cm]{#1}} ;}] (0,1) circle (1) ;\end{figure}}



%--------------------------------------------------------------------------------------------
% FOOTER
%--------------------------------------------------------------------------------------------
\usepackage{lastpage}
\usepackage{fancyhdr} % http://mirror.easyname.at/ctan/macros/latex/contrib/fancyhdr/fancyhdr.pdf
 
\fancypagestyle{pagenumbering}{
\fancyhf{}
\renewcommand{\headrulewidth}{0pt}
% add the page count in the middle: \thepage gives you the current page, \pageref{LastPage} the total
\cfoot{\thepage/\pageref{LastPage}}
\lfoot{\color{black!40}\textsc{Article $\cdot$ Outline $\cdot$ Template}}
\rfoot{\vspace{-10mm}\includegraphics[width=0.1\textwidth]{logo.png}} % \roundpic{ninja.png}
}%


\title{Article Outline Template}
\author{Sarah Lang}
\date{June 2020}



\begin{document}
\pagestyle{pagenumbering}


\begin{quote}
    \underline{Act and reflect -- but never at the same time! (\emph{Steven Pressfield})} \\[0.5em]
    \textbf{Act:} Free writing (given previous knowledge of what you want to write about \\
    \& after having written the outline. No re-reading or perfecting sentences allowed.\\
    \textbf{Reflect:} Add references, edit the text, re-read. 
\end{quote}

\section{Brainstorming Phase}

\begin{minipage}[t]{0.45\textwidth}

\subsection*{\MakeUppercase{Abstract Trias}}
\textbf{1. Topic and its relevance to the field (broad)}
\vspace{3cm}

\textbf{2. Yet unsolved pressing problem you discovered} (Hero Narrative!)
\vspace{3cm}

\textbf{3. Your solution to the problem}
\vspace{3cm}

\textbf{(4.) Conclusion\dots }

{\scriptsize This reseach will show/contribute XY to the discussion/field of AB (relevance again, what are the important outcomes of this research

}
\vspace{2cm}

\end{minipage}\hfill
\begin{minipage}[t]{0.5\textwidth}

%\fcolorbox{black}{white}{ \color{white} {\lipsum[55] } }
\begin{tcolorbox}\subsection*{\MakeUppercase{Research Question}}\color{white}\lipsum[75]\end{tcolorbox}
\begin{tcolorbox}\subsection*{\MakeUppercase{Hypothesis}}\color{white}\lipsum[75]\end{tcolorbox}
\subsection*{Most relevant literature to include}
\vspace{2cm}

{\scriptsize $\to$ focus on essentials only! name just a few items (no web-searching Google scholar again, you already know what needs to be in there because you have completed the research phase -- now is the phase where you just write it down, not do new research, remember?)

}

\end{minipage}\hfill
\begin{minipage}[t]{0.45\textwidth}

\subsection*{\MakeUppercase{Formal requirements}}
\textbf{Word count}\\[1em]

\textbf{Intended Audience}
{\scriptsize what are this audience's main interests in your topic? What might you focus on more to please them?

}\vspace{2cm}

\textbf{Intended Journal}
{\scriptsize as above but also: citation style, formal requirements

}
\vspace{2cm}


\begin{tcolorbox}\subsection*{\MakeUppercase{if it's part of a book}}
{\scriptsize where is it placed? -- do the chapter create a continuous argument?

}\vspace{0.5em}

\textbf{Chapter before (main idea)}\\[1cm]
\textbf{Current chapter to write (main idea)}\\[1cm]
\textbf{Chapter after (main idea)}\\[1cm]

\end{tcolorbox}

\end{minipage}\hfill
\begin{minipage}[t]{0.5\textwidth}

%\fcolorbox{black}{white}{ \color{white} {\lipsum[55] } }
\begin{tcolorbox}\subsection*{\MakeUppercase{Keywords / Buzzwords}}{\scriptsize be sure to use relevant keywords so your paper will be found

}

\color{white}\lipsum[75]\end{tcolorbox}
\begin{tcolorbox}\subsection*{\MakeUppercase{Title}}{\scriptsize make sure the title contains those buzzwords and be as concrete as possible about what the paper is about (if relevant, include topic, place, time, material, method and focus/issue at hand), in an immediately understandable way. Don't make it too long but rather include one more keyword if unsure

}\vspace{0.5em}

\color{white}\lipsum[75]\end{tcolorbox}
\begin{tcolorbox}\subsection*{\MakeUppercase{SCHEDULE}}
\footnotesize
\textbf{Word count required: }

{\scriptsize will often be 7000-10.000 words for a scholarly article of 10--15 pages

}\vspace{1em}

\textbf{No. of 1h time blocks for free writing: }

{\scriptsize a 1h free writing session easily yields 1000 words of raw draft

}\vspace{2em}

\subsection*{\MakeUppercase{Sessions: What? When?}}
\begin{enumerate}
    \item \textbf{\lbrack{}rightaway\rbrack{}} Outline and Intro/Abstract
    \item 
    \item 
    \item 
    \item 
    \item 
    \item 
    \item
\end{enumerate}\vspace{2em}

\hrule

\begin{itemize} {\bf
    \item 1 session backing up with references
    \item 2 sessions editing the text (flow, logic of argument, etc.)
    \item 1 session of proof-reading }
    {\scriptsize probably around 1h per session depending how thoroughly you do it}
\end{itemize}
\end{tcolorbox}

\end{minipage}


\clearpage
\section{Outline}

\begin{tcolorbox}
\subsection*{\MakeUppercase{Intro}}
{\scriptsize Write out the abstract trias as an abstract / introductory sentence. All this information (and maybe also all keywords / key pieces of info like when, where what data from the title are adressed so people immediately have a clear picture what it's about and what the goals are.

Outline the order of the argument: How are you going to show the problem and the solution? What methods are used? What are the materials?}
\color{white}\lipsum[75]\end{tcolorbox}


\begin{tcolorbox}\subsection*{\MakeUppercase{Topic \& Literature Review}}\color{white}\lipsum[75]\end{tcolorbox}

\begin{tcolorbox}\subsection*{\MakeUppercase{Materials}}
\begin{itemize}\footnotesize
    \item Sources / Examples
    \item 1-2 images for visualiation purposes
    {\scriptsize How do they figure into the argument?
    
    }
\end{itemize}
\color{white}\lipsum[75]\end{tcolorbox}

\begin{tcolorbox}\subsection*{\MakeUppercase{Methods}}\color{white}\lipsum[75]\end{tcolorbox}

\begin{tcolorbox}\subsection*{\MakeUppercase{The Problem Explained/Shown}}{\scriptsize Show, don't tell!}
\color{white}\lipsum[75]\end{tcolorbox}
\begin{tcolorbox}\subsection*{\MakeUppercase{The Solution Demonstrated}}\color{white}\lipsum[55]\end{tcolorbox}

\begin{tcolorbox}\subsection*{\MakeUppercase{Conclusion}}
\begin{enumerate}\footnotesize
    \item What has been done? (recap intro / abstract)\\[0.5em]
    \item What was the problem and how was it solved?\\[0.5em]
    \item What was demonstrated using the given method and why did that make sense?\\[0.5em]
    \item Repeat the results very clearly\\[0.5em]
    \item What new questions came up? -- What is still open for further research?\\[0.5em]
    \item Final sentence: What big relevant area did this contribute new important perspectives to?\\[1em]
\end{enumerate}\end{tcolorbox}


\end{document}
